% camiregu 2024-jan-14
\chapter{Experimental setup}

%----------------------------------------------------------------------------------------

\section{Apparatus}
The instruments used were as follows:

\begin{lrapparatus}
    \item A lamp
    \item A mirror
    \item A screen
    \item A biconvex BK7 glass lens with refractive index $1.51502$
    \item A metre stick ($\pm \qty{0.5}{\milli\metre}$)
    \item A ruler ($0 \text{ to } 30\unit{\centi\metre}$) ($\pm \qty{0.5}{\milli\metre}$)
    \item Two vertex pointers of length $(152.47 \pm 0.02) \unit{\milli\metre}$
    \item A G\&G Spherometer ($-8 \text{ to } 8 \unit{\milli\metre}$) ($\pm \qty{25}{\micro\metre}$) with a leg-to-screw distance of $\qty{22.5}{\milli\metre}$ (measured by ruler)
\end{lrapparatus}

\section{Procedure}

\paragraph{Distant Object}
For this part of the procedure, we used the ceiling lights as our object, estimating them to be about $2.5 \unit{\metre}$ above the desk, which we used as our screen. A lens was held above the desk until the image of the ceiling lights was in sharp focus, at which point the height of the lens was measured. This was repeated over 5 trials.

\paragraph{Mirror Method}
Next, the lens was secured onto a flat surface to ensure the precision of the optical axis. A lamp, our object, was placed to the left, and a mirror to the right (as in \cref{fig:infinityimage}). The distance between the object and the lens was then adjusted until the image was in sharp focus, at which point all distances were measured. This was repeated over 5 trials.

\lrfigure[htbp]{infinityimage}{Experimental setup for the mirror method portion of the experiment}

\paragraph{Thin Lens (Imprecise)}
At this point, the mirror was replaced by a screen (as in \cref{fig:thinlens}), and the lens and screen position were adjusted until the image was in sharp focus, at which point all distances were measured. This was repeated over 10 trials.

\lrfigure[htbp]{thinlens}{Experimental setup for the imprecise thin lens portion of the experiment}

\paragraph{Thin Lens and Magnification}
First, the height of the object was measured with a ruler. To add a final layer of precision to our previous setup, two vertex pointers were used to measure distance. One was placed between the lamp and the lens, the other between the lens and the screen. Once again, the lens and the screen's positions were adjusted until the image was in sharp focus. At this point, four lateral measurements were taken: the position of the left vertex pointer touching the lamp; the position of the left vertex pointer touching the lens; the position of the right vertex pointer touching the lens; and the position of the right vertex pointer touching the screen. This setup is illustrated in \cref{fig:vertexpointer}. In addition to these, the height of the image was also measured using a ruler.

\lrfigure[htbp]{vertexpointer}{Experimental setup for the thin lens and magnification portions of the experiment}

\paragraph{Lensmaker}
First, the distance between the central screw of the spherometer and the leg of the spherometer was measured using a ruler. Then the spherometer was placed atop one face of the lens, and the screw's vertical position adjusted until it touched the face of the lens. This spherometer reading was recorded, and the procedure was repeated on the lens' second side. The entire procedure was then repeated over 3 trials.