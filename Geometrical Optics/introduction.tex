% camiregu 2024-jan-14
\chapter{Introduction}

%----------------------------------------------------------------------------------------
\section{Objectives}
This experiment was designed to explore a variety of different techniques that can be used to measure the focal length of a spherical lens. In the following pages we describe how theoretical equations were manipulated to make focal length depend on different measurable quantities, the different methods created for measuring those quantities, and how well those methods compare with each other.

In exploring the following techniques, we also gained a deeper intuitive understanding of the geometrical optics principles underlying the techniques. Namely, the laws of reflection, refraction, and magnification.

\section{Thin Lens Equation}
First, we use the "thin lens" equation:

\[\frac{1}{f} = \frac{1}{s_o} + \frac{1}{s_i}\]

which tells us the relation between the focal length of a lens $f$, the distance of the object from the lens along the optical axis $s_o$, and the distance of the image along the optical axis $s_i$. Note that this equation assumes all rays are paraxial, and that the lens is spherical and of negligible thickness. Importantly, for an object on the optical axis, $s_i$ is where the image will be in focus.

Then perhaps the most straightforward approach to measure the focal length is to place an object along the optical axis and measure where the image is in focus. In order to make use of these quantities, we rearrange the thin lens equation to find:

\lrequation{thinlens}{f}
{2}{\frac{#1 #2}{#1 + #2}}
{{s_o}{s_i}}

Alternatively, using many trials, we may plot $\frac{1}{s_o}$ against $\frac{1}{s_i}$ and find the focal length at the $y$-intercept, that is, as $\frac{1}{s_i} \to 0$. This concept is exploited further in the next section.

\subsection{Limits at Infinity}
There is yet more we can do. Note that in the thin lens equation, if $s_o \to \infty$, then that respective term goes to zero, and we obtain an extremely simple relation for $f$. Namely,

\begin{align*}
    \frac{1}{f} &= \lim_{s_o \to \infty} \left(\frac{1}{s_o} + \frac{1}{s_i}\right) \\
    &= \frac{1}{s_i},
\end{align*}

or equivalently,

\lrequation{infinityobject}{f}
{1}{#1}
{{s_i}}

A similar process with $s_i \to \infty$ gives us

\lrequation{infinityimage}{f}
{1}{#1}
{{s_o}}

Finally, we exploit the concept of an "object at infinity" by noting that it is equivalently an object emanating parallel rays of light. Thus we can approximate an object at infinity by a distant object, and simulate an image at infinity by a mirror.

\subsection{Using Magnification}
For this method we use what we know about magnification under the previously stated approximations, namely,

\begin{equation} \label{eqn:magnification}
    M = \frac{h_i}{h_o} = -\frac{s_i}{s_o},
\end{equation}

which we combine with the thin lens equation to find

\lrequation{magnificationgraph}{\frac{1}{M}}
{1}{-\frac{#1}{f} + 1}
{{s_o}}

So may plot $\frac{1}{M}$ against $s_o$ to find $f$ as the slope.

\section{Lensmaker Equation}
The lensmaker equation gives us a way to find the focal point of a double lens based on its geomtry. In particular,

\[ \frac{1}{f} = \left(n-1\right)\left(\frac{1}{R_1} - \frac{1}{R_2}\right) \]
\[ \Downarrow \]

\lrequation{lensmaker}{f}
{3}{\frac{#1 #2}{\left(#2 - #1\right)\left(#3 - 1\right)}}
{{R_1}{R_2}{n}}

where $R_1, R_2$ are the radii of curvature of each side of the lens, and $n$ is the refractive index of the lens' material. Using a spherometer, these radii are found by

\lrequation{spherometer}{R}
{2}{\frac{#1^2 + #2^2}{2#1}}
{{r}{d}}

where $r$ is the spherometer reading, and $d$ is the distance from the leg of the spherometer to the screw. So by simply taking some readings of the lens using the spherometer, we are able to find its focal length.