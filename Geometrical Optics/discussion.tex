% camiregu 2024-jan-14
\chapter{Discussion}

%----------------------------------------------------------------------------------------

Technically, the most accurate technique was the \textbf{Distant Object} technique, yielding $f = (195 \pm 17) \unit{\milli\metre}$, though it was extremely imprecise. Both the \textbf{Mirror} and \textbf{Thin Lens (Individual)} methods were extremely precise, and only slightly less accurate. The mirror method in particular is notable for how simple and quick it was to set up.

The \textbf{Magnification} technique was reasonably effective, as was the \textbf{Lensmaker} technique, but the graphical \textbf{Thin Lens} method was uniquely ineffective at $f = (210 \pm 180) \unit{\milli\metre}$ being both extremely imprecise, and the least accurate. It also was the part of the experiment that took us the most time to perform, \textit{and} had a more complicated analysis. As this was based on the graph's $y$-intercept, that is, when $\frac{1}{s_o} \to 0$, it is even worse considering how much more effective the distant object technique was with the same theoretical justification and less precise instruments.

Aside from this, one notable flaw in the distant object technique that was not properly accounted for, was how imprecise the readings taken of the image distance were. In real-time, hands were wobbling, and the smallest division of the ruler was certainly not the smallest division we were able to differentiate; as such, the reading error should have been much higher. The experiment also required us to simply estimate the height of the ceiling, which is totally arbitrary and drastically affects the systematic error. While we could have also estimated a higher reading error, there is no reason to introduce such arbitrary "measurements" into our labs unless it's strictly necessary. The first problem would have been easily solved with a retort stand; the second, with a measuring tape and a ladder.